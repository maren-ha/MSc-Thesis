\chapter*{Zusammenfassung in deutscher Sprache}

%Biomedizinische Zeitreihendaten von Individuen sind oft durch ein dünn besetztes und unregelmäßiges Zeitraster von Messungen sowie durch unterschiedliche individuelle Entwicklungsmuster gekennzeichnet. Motiviert durch ein Szenario aus der epidemiologischen Studie der Deutschen Nationalen Kohorte (NAKO) und inspiriert von den jüngsten Fortschritten bei der Kombination von Black-Box-Deep Learning mit expliziter mechanistischer Modellierung durch Differentialgleichungen, entwickeln wir ein generatives Modell, das die individuelle Dynamik in einer niedrigdimensionalen latenten Darstellung als Lösungen von gewöhnlichen Differentialgleichungen (ODEs) erfasst.  
%Trotz des wachsenden Potenzials und der Nachfrage nach personalisierter Medizin fehlt es häufig an geeigneten Daten. Eine individualisierte Behandlung erfordert eine Charakterisierung der dynamischen Entwicklung jedes einzelnen Individuums im Laufe der Zeit, aber die Erfassung dichter Zeitreihendaten mit vielen Variablen, die zu mehreren Zeitpunkten gemessen werden, ist aufwändig. 

Bei der Betrachtung biomedizinischer Zeitreihendaten ist das Verständnis der zugrunde liegenden dynamischen Systeme auf individueller Ebene der erste und wesentliche Schritt für eine personalisierte Behandlung. Solche Daten weisen jedoch oftmals ein spärliches und unregelmäßiges Zeitraster von Messungen sowie individuell verschiedene Entwicklungen auf, was die Modellierung erschwert. 

In dieser Arbeit wird basierend auf einem generativen Deep Learning-Verfahren ein Modell entwickelt, das aus solchen spärlichen und unregelmäßig beobachteten Zeitreihendaten einen niedrigdimensionalen latenten Raum lernt, in dem die individuellen Entwicklungsmuster als Lösungen gewöhnlicher Differentialgleichungen repräsentiert werden. Meine Methodik ist dabei von aktuellen Forschungsarbeiten zur Kombination von Deep Learning-Verfahren mit einer expliziten mechanistischen Modellierung durch Differentialgleichungen inspiriert und durch ein Szenario aus der Nationalen Kohorte (NAKO), einer groß angelegten epidemiologischen Kohortenstudie, motiviert.

Basierend auf einer NAKO-Substudie simuliere ich Daten, die sich durch eine umfangreiche Charakterisierung jedes Individuums mit Messungen vieler Variablen zu einem Baseline-Zeitpunkt auszeichnen, wobei eine kleinere Teilmenge dieser Variablen zu einem zweiten, für jedes Individuum unterschiedlichen, Zeitpunkt erneut gemessen wird. Insgesamt liegen somit individuelle dynamische Prozesse vor, die nur selten (je zwei Zeitpunkte pro Individuum) und unregelmäßig mit einer gewissen Messunsicherheit beobachtet werden. 
Ziel meiner Arbeit ist es, in einem solchen Szenario ein Modell zu entwickeln, das trotz der Messfehler sowie des groben und unregelmäßigen Zeitrasters die zugrunde liegenden individuellen Entwicklungsmuster extrahieren kann. 

Hierbei verwende ich einen Variational Autoencoder (VAE), ein generatives Deep-Learning-Modell, um anhand der beobachteten Entwicklungsmuster in einem nichtlinearen, unüberwachten Lernverfahren einen niedrigdimensionalen latenten Raum abzuleiten, der die zentrale, den Daten zugrunde liegende Dynamik repräsentiert. Um glatte Trajektorien zu modellieren, wird der latente Raum auf einen Raum differenzierbarer Funktionen eingeschränkt, die als Lösungen eines vorab definierten ODE-Systems vorliegen. Basierend auf der Annahme, dass individuelle Unterschiede in den Entwicklungsmustern auf Unterschiede in den nur bei zum Baseline-Zeitpunkt gemessenen Variablen zurückgeführt werden können, verwende ich diese Baseline-Variablen, um mithilfe eines weiteren neuronalen Netzes individuelle ODE-Parameter zu bestimmen.

Um komplexere zugrunde liegende dynamische Systeme modellieren zu können, erweitere ich das Modell und bestimme für jedes Individuum eine Gruppe von Individuen mit ähnlichen Entwicklungsmustern. Die Kombination aller zweiten Messungen in der Gruppe dient dann als Proxy-Information für das betrachtete Individuum über die gemeinsame dynamische Entwicklung zu mehreren Zeitpunkten. Durch das Trainieren des Modells auf diesen Gruppen wird so die Unregelmäßigkeit der zweiten Messzeitpunkte ausgenutzt und die Information jedes einzelnen Individuums wird um zusätzliche, stellvertretende Informationen von anderen Individuen erweitert.
% und somit das Problem des dünnbesetzen Zeitgitters zu adressieren. 

In den Anwendungen auf die oben beschriebenen simulierte Daten konnte ich zeigen, dass das entwickelte Modell in der Lage ist, individuelle Entwicklungsmuster basierend auf linearen und nicht-linearen zweidimensionalen ODE-Systemen mit zwei oder vier unbekannten Parametern zu rekonstruieren und für ein gegebenes Individuum eine Gruppe von Individuen mit ähnlichen Entwicklungen zu ermitteln. 

Zusammenfassend bietet die vorgestellte Methode somit auf individueller Ebene Einsicht in die dynamischen Systeme, die den Entwicklungen verschiedener Individuen im Zeitverlauf zugrunde liegen, und kann die Planung personalisierter Interventionen anhand der Kenntnis solcher vollständigen individualspezifischen dynamischen Systeme ermöglichen.